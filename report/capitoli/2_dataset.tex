\section{Introduzione}
In questo capitolo, nella sezione \textbf{\ref{sec:descrizione}} vengono elencati e descritti in generale i dataset presi in esame, mentre nella sezione \textbf{\ref{sec:profiling}} viene fatto del data profiling per poter ottenere qualche insight sui dati, in particolare nella sottosezione \textbf{\ref{sec:cleaning}} i dati vengono puliti e preparati all'uso.\\
Infine, nella sezione \textbf{\ref{sec:considerazioni}} vengono tratte alcune considerazioni su tutti i dataset.

\section{Descrizione} \label{sec:descrizione}
Per l'esperimento sono stati selezionati alcuni dataset di time series dal sito \url{http://www.timeseriesclassification.com/dataset.php} in formato TSV (Tab Separated Value).\\
Ciascuna TS ha una \textbf{label} associata, corrispondente ad una classe di riferimento assegnata dagli autori dei dataset.\\
Sono stati scelti di dimensioni e caratteristiche differenti l'uno dall'altro:
\begin{itemize}
	\item \textbf{ECG5000}, contenente 5000 elettrocardiogrammi (ECG) ottenuti da pazienti affetti da insufficienza cardiaca. Le cinque classi sono state assegnati per annotazione automatica dagli autori;
	\item \textbf{ECG200}, contenente 200 elettrocardiogrammi ottenuti da pazienti affetti da insufficienza cardiaca. Le due classi rappresentano gli ECG di soggetti sani e di soggetti affetti da infarto al miocardio;
	\item \textbf{ChlorineConcentration}, contenente dati relativi alla presenza di cloro nell'acqua, recuperati usando 166 sensori lungo una rete idraulica per 15 giorni;
	\item \textbf{FordA}, contenente dati relativi al rumore di motori di autovetture ottenuti sotto normali condizioni d'uso;
	\item \textbf{FordB}, contenente dati relativi al rumore di motori di autovetture, alcuni sotto normali condizioni d'uso, altri sotto condizioni di elevato rumore;
	\item \textbf{PhalangesOutlinesCorrect}, contenente dati relativi ad outline di falangi ottenuti usando algoritmi di estrazione di outline da immagini di radiografie delle mani. Questi outline possonoe essere corretti o errati;
	\item \textbf{RefrigerationDevices}, contenente dati relativi al consumo di tre diversi dispositivi di refrigerazione domestici;
	\item \textbf{TwoLeadECG}, contenente elettrocardiogrammi di due tipi diversi di segnale;
	\item \textbf{TwoPatterns}, contenente TS generate artificalmente, ciascuna caratterizzata da una coppia di pattern nel segnale: up-up, up-down, down-up e down-down.
\end{itemize}

\begin{figure}[H]
	\centering
	\makebox[0pt]{\includegraphics[width=1.2\paperwidth]{ecg200.png}}
	\caption{Plot di alcune TS classificate in ECG200. Si nota facilmente che, sebbene classificate insieme, le TS della classe 2 non presentano sembrano avere andamenti simili.}
	\label{fig:ecg200}
\end{figure}

Di seguito sono riportate le caratteristiche principali dei dataset:
\begin{table}[H]
	\centering
	\begin{tabularx}{\textwidth}{X c c c c}
		\hline
		\textbf{Name} & \textbf{Train size} & \textbf{Test size} & \textbf{Sequence length} & \textbf{Classes} \\
		\hline
		\textbf{ECG5000} & 500 & 4500 & 140 & 5\\
		\textbf{ECG200} & 100 & 100 & 96 & 2\\
		\textbf{Chlorine} & 476 & 3840 & 166 & 3\\
		\textbf{FordA} & 3601 & 1320 & 500 & 2\\
		\textbf{FordB} & 3636 & 810 & 500 & 2\\
		\textbf{Phalanges} & 1800 & 858 & 80 & 2\\
		\textbf{Refrigeration} & 375 & 375 & 720 & 3\\
		\textbf{TwoLeadECG} & 23 & 1139 & 82 & 2\\
		\textbf{TwoPatterns} & 1000 & 4000 & 128 & 4\\
	\end{tabularx}
	\caption{Caratteristiche principali dei dataset (i nomi sono stati abbreviati per una migliora leggibilità).}
	\label{tab:datasets}
\end{table}

\section{Data Profiling} \label{sec:profiling}
Attraverso la libreria Python \textit{Pandas} è stato possibile fare un po' di data profiling, andando ad osservare nel dettaglio la struttura di questi dataset.\\
Sono emerse alcune informazioni interessanti:
\begin{itemize}
	\item Tipicamente le label sono valori interi non negativi, che partono da 0 o da 1, ma in alcuni dataset sono state usate \textbf{label negative};
	\item Le classi di alcuni dataset sono \textbf{sbilanciate}, ad esempio in ECG5000 ci sono 2919 sample nella classe 1, mentre solo 24 nella classe 5;
	\item Tutte le \textbf{time series sono della stessa lunghezza};
	\item \textbf{Ciascuna osservazione è un singolo valore scalare}, quindi i dataset hanno profondità pari a 1;
	\item Non ci sono \textbf{valori nulli}, quindi tutte le osservazioni sono definite;
	\item Sono stati identificati molti valori \textbf{outlier}, infatti alcune colonne hanno una distrubuzione di valori molto più ampia rispetto alle altre, probabilmente dovuti ad errori di misurazione dei sensori;
	\item Le dimensioni di training e test set sono sbilanciate e con una cattiva distribuzione delle TS rispetto alle diverse classi.
\end{itemize}
Lo sbilanciamento è stato individuato facendo il \textbf{raggruppamento rispetto alla colonna 0}, ovvero quella relativa alle label assegnate alle TS, mentre gli outlier sono stati individuati \textbf{guardando la media, la deviazione standard e i quartili di ciascuna colonna}, oltre che il plot dei dati.\\
L'assenza di valori null è stata riscontrata grazie al conteggio dei valori null di Pandas.\\
\\
Di seguito un riassunto della distribuzione dei valori nelle classi di ciascun dataset (il numero associato ad ogni classe non rispecchia l'effetivo valore delle label):
\begin{table}[H]
	\centering
	\begin{tabularx}{\textwidth}{X c c c c c}
		\hline
		\textbf{Name} & \textbf{Class 1} & \textbf{Class 2} & \textbf{Class 3} & \textbf{Class 4} & \textbf{Class 5} \\
		\hline
		\textbf{ECG5000} & 2919 & 1767 & 96 & 194 & 24\\
		\textbf{ECG200} & 133 & 67 & / & / & /\\
		\textbf{ChlorineConc.} & 1000 & 1000 & 2307 & / & /\\
		\textbf{FordA} & 2394 & 2527 & / & / & /\\
		\textbf{FordB} & 2185 & 2261 & / & / & /\\
		\textbf{Phalanges} & 960 & 1698 & / & / & /\\
		\textbf{Refrigeration} & 250 & 250 & 250 & / & /\\
		\textbf{TwoLeadECG} & 581 & 581 & / & / & /\\
		\textbf{TwoPatterns} & 1306 & 1248 & 1245 & 1201 & /\\
	\end{tabularx}
	\caption{Distribuzione delle classi dei dataset}
	\label{tab:labels}
\end{table}
Com'è possibile notare dalla \figurename~\ref{fig:ecg200}, alcuni dataset presentano una classificazione che non è basata sull'andamento dei dati nel tempo, bensì su altre caratteristiche dei dati stessi (nel caso di ecg200 se i pazienti fossero sani o infartuati).\\
Questo implica necessariamente che un clustering effettuato basandosi unicamente sull'andamento della TS non potrà mai coincidere con una classificazione del genere.\\
Questo aspetto verrà tenuto conto durante le valutazioni dei risultati.

\subsection{Data cleaning and preparation} \label{sec:cleaning}
In risposta ai problemi individuati durante la fase di data profiling, sono state adottate alcune azioni correttive:
\begin{itemize}
	\item Tutte \textbf{le label sono state modificate} in modo tale da essere degli \textbf{interi positivi che partono da 1 e che si incrementano unariamente}, rispecchiando la numerazione presente nella tabella \ref{tab:labels}.\\
	Questa pulizia è stata fatta direttamente sui file dei dataset;
	
	\item Il problema dello sbilanciamento delle classi non poteva essere risolto poiché è un elemento intrinseco dei dataset. Tuttavia, è stato possibile risolvere il problema degli outlier, \textbf{"tagliando" tutte le osservazioni che andavano al di sotto del 3-percentile oppure oltre il 97-percentile} e ricondurle a loro.\\
	Questa pulizia viene fatta ogni volta che viene caricato un dataset in memoria.
	
	\item Il problema dello sbilanciamento di training e test set è stato risolto \textbf{unendo i due insiemi in memoria e poi dividerli casualmente con la regola dell'80/20\footnote{80\% dei dati è di train, mentre il restante 20\% è di test.} di default}.\\
	Questa operazione viene fatta ogni volta che viene caricato un dataset in memoria.
\end{itemize}

\section{Considerazioni} \label{sec:considerazioni}
Lo \textbf{sbilanciamento delle classi potrebbe causare qualche problema nella fase di valutazione esterna del clustering}, poiché diventa difficile per k-Means creare cluster molto piccoli se i dati sono effettivamente molto simili tra loro.