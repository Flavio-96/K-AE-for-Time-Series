\section{Descrizione del modello}
Cos'è un autoencoder
Casi d'uso tipici
Perché lo abbiamo usato noi in questo caso
Come lo abbiamo usato (setting iperparametri, cosa abbiamo preso da esso (i vettori latenti) e cosa no)

\subsection{Training}
Come è stato allenato il modello (tensor flow e l'idea descritta con snippet notebook)

\section{Clustering con k-Means}
Come è stato lanciato k-means (su quali dati e con quali iperparametri)
Come è stato determinato il k, e per quali k abbiamo provato
Perché non si usa il DTW: i latent vector non sono TS!!

\subsection{Valutazione dei risultati}
Descrizione metriche, interne ed esterne, per ogni dataset
Osservazioni dei risultati (ad es. la purity è spesso alta, mentre un'altra cosa varia molto, o che su dataset grandi funziona bene, su piccoli no, ecc.): risultati da importare da file perché scriverli a mano è nu burdell.
-------------------------------------\\

Sia n il numero di classi di un dataset:
\begin{enumerate}[(i)]
	\item Se $n=2$, allora si esegue 2-Means, 3-Means e 4-Means;
	\item Se $n>2$, allora si esegue n-1-Means, n-Means ed n+1-Means.
\end{enumerate}

Di seguito sono riportati i risultati del clustering di alcuni dei dataset scelti.
\begin{table}[H]
	\centering
	\begin{tabularx}{\textwidth}{X | c c c c}
		\hline
		\textbf{Name} & \textbf{\#clusters} & \textbf{Silhouette} & \textbf{DB} & \textbf{Dunn} \\
		\hline
		\textbf{ECG5000} & 4 & \textbf{0.299} & \textbf{1.583} & / \\
		& 5 & 0.184 & 1.915 & / \\
		& 6 & 0.174 & 1.748 & / \\
		\hline
		\textbf{ECG200} & 2 & 0.220 & 1.215 & / \\
		& 3 & \textbf{0.326} & \textbf{1.059} & / \\
		& 4 & 0.277 & 1.270 & / \\
		\hline
		\textbf{ChlorineConc.} & 2 & \textbf{0.437} & 0.970 & / \\
		& 3 & 0.394 & 0.990 & / \\
		& 4 & 0.389 & \textbf{0.892} & / \\
		\hline
		\textbf{FordA} & 2 & \textbf{0.02} & \textbf{6.873} & / \\
		& 3 & 0.005 & 9.193 & / \\
		& 4 & -0.001 & 10.972 & / \\
	\end{tabularx}
	\caption{Misure interne del clustering con 100 iterazioni}
	\label{tab:clustering100_int}
\end{table}

\begin{table}[H]
	\centering
	\begin{tabularx}{\textwidth}{X | c c c c c}
		\hline
		\textbf{Name} & \textbf{\#clusters} & \textbf{Purity} & \textbf{Rel. Purity} & \textbf{ARI} & \textbf{FM}\\
		\hline
		\textbf{ECG5000} & 4 & 0.885 & \textbf{0.692} & \textbf{0.494} & \textbf{0.690} \\
		& 5 & 0.888 & 0.526 & 0.395 & 0.614 \\
		& 6 & \textbf{0.920} & 0.606 & 0.409 & 0.623 \\
		\hline
		\textbf{ECG200} & 2 & 0.688 & 0.688 & 0.030 & \textbf{0.709} \\
		& 3 & \textbf{0.75} & \textbf{0.75} & \textbf{0.230} & 0.627 \\
		& 4 & \textbf{0.75} & \textbf{0.75} & 0.108 & 0.468 \\
		\hline
		\textbf{ChlorineConc.} & 2 & \textbf{0.533} & 0.25 & \textbf{-0.001} & \textbf{0.444} \\
		& 3 & \textbf{0.533} & 0.25 & \textbf{-0.001} & 0.380 \\
		& 4 & \textbf{0.533} & \textbf{0.341} & \textbf{-0.001} & 0.315 \\
		\hline
		\textbf{FordA} & 2 & 0.522 & 0.522 & \textbf{0.001} & \textbf{0.5} \\
		& 3 & \textbf{0.528} & \textbf{0.528} & \textbf{0.001} & 0.419 \\
		& 4 & \textbf{0.528} & \textbf{0.528} & \textbf{0.001} & 0.364 \\
	\end{tabularx}
	\caption{Misure esterne del clustering con 100 iterazioni}
	\label{tab:clustering100_ext}
\end{table}